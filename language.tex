\documentclass{memoir}

\usepackage{tipa}
\usepackage{makeidx}
\makeindex

\title{Language X}
\author{Aaron J. Lang}

\let\ipa\textipa{}

\newcommand{\wt}[1]{\textipa{/#1/}}
\newcommand{\nt}[1]{\textipa{[#1]}}
\newcommand{\ot}[1]{$\langle$#1$\rangle$}

\begin{document}
\frontmatter
\maketitle
\newpage
\tableofcontents
\mainmatter{}

\part{Grammar}
\chapter{Phonology}
\index{Phonology}

\section{Phoneme Inventory}

\begin{table}[h]
\centering
\begin{tabular}{*{6}{c}}
  m   &     &           & n   &           & \ipa{N} \\
  p b &     &           & t d &           & k g     \\
      & f v & \ipa{T D} & s z & \ipa{S Z} & x       \\
      &     &           & l \ipa{\textltilde} \\
      &     &           & \ipa{\*r R r} \\
\end{tabular}
\caption{Consonants}
\end{table}

\begin{tabbing}
  \wt{Z} \= sounds like \= `plea\underline{s}ure' \= and can be written \= \ot{zh} \kill
  \wt{N} \> sounds like \> `si\underline{ng}'     \> and can be written \> \ot{nh} \\
  \wt{T} \> sounds like \> `\underline{th}is'     \> and can be written \> \ot{th} \\
  \wt{D} \> sounds like \> `\underline{th}is'     \> and can be written \> \ot{dh} \\
  \wt{S} \> sounds like \> `\underline{sh}in'     \> and can be written \> \ot{sh} \\
  \wt{Z} \> sounds like \> `plea\underline{s}ure' \> and can be written \> \ot{zh} \\
  \wt{x} \> sounds like \> `lo\underline{ch}'     \> and can be written \> \ot{kh} \\
\end{tabbing}

%l is pronounced as one of [l,ɹ,ɾ,r] depending on the surrounding sounds,
%and can be written as <l,r\,4,r> or <l,rh,rr,r> respectively

\begin{table}[h]
\centering
\begin{tabular}{*{3}{c}}
       i  &         &      u  \\
  \ipa{I} &         & \ipa{U} \\
       e  &         &      o  \\
  \ipa{E} &         & \ipa{O} \\
       a  & \ipa{A} & \ipa{6} \\
\end{tabular}
\caption{Vowels}
\end{table}

\clearpage % hard formatting

\begin{tabbing}
  \wt{i} \= sounds like \= `pea' \= and can be written \= \ot{zh} \kill
  \wt{i} \> sounds like \> `pea' \\
  \wt{u} \> sounds like \> `poo' \\
  \nt{I} \> sounds like \> `pit' \> and can be written \> \ot{ih} \\
  \nt{U} \> sounds like \> `put' \> and can be written \> \ot{uh} \\
  \wt{e} \> sounds like \> `pay' \\
  \wt{o} \> sounds like \> `oat' \\
  \wt{E} \> sounds like \> `pet' \> and can be written \> \ot{eh} \\
  \wt{O} \> sounds like \> `oar' \> and can be written \> \ot{ohh} \\
  \wt{a} \> sounds like \> `pat' \\
  \wt{A} \> sounds like \> `pot' \> and can be written \> \ot{oh} \\
  \wt{6} \> sounds like \> `par' \> and can be written \> \ot{ahh} \\
\end{tabbing}

\section{Allophony}

/l/

\section{Accents}

Some accents contain ejectives

Some exhibit velarisation and/or palatisation

\section{Phonotactics}

Velars seem to appear predominantly as codas

Voiced stops rarely occur in a coda position

Nuclei can be composed of vowels, nasals, fricatives (without secondary articulation), and /l/

Single syllable morphemes require a vowel or fricative nucleus

% Misc
% ----
% The lexicon appears to be prefix free?

\part{Lexicon}
\index{Lexicon}

\part{Corpus}

\backmatter
\printindex
\end{document}
