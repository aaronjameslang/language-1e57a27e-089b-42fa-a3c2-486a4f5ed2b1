\documentclass{memoir}

\usepackage{makeidx}
\makeindex

\title{Language X}
\author{Aaron J. Lang}

\begin{document}
\frontmatter
\maketitle
\newpage
\tableofcontents
\mainmatter

\part{Grammar}
\chapter{Phonology}
\index{Phonology}

\section{Phononeme Inventory}

| m   |     |     | n   |     | ŋ   |
| p b |     |     | t d |     | k g |
|     | f v | θ ð | s z | ʃ ʒ | x   |
|     |     |     | l   |     |     |

ŋ can be written <N> or <nh>, and sounds like "si*ng*"

θ can be written <T> or <th>, and sounds like "*th*in"

ð can be written <D> or <dh>, and sounds like "*th*is"

ʃ can be written <S> or <sh>, and sounds like "*sh*in"

ʒ can be written <Z> or <zh>, and sounds like "plea*s*ure"

l is pronounced as one of [l,ɹ,ɾ,r] depending on the surrounding sounds,
and can be written as <l,r\,4,r> or <l,rh,rr,r> respectively

    | Front                           || Back                            |
    | /i/ | [i,ɪ] | <i,I> | "pea,bit" || /u/ | [u,ʊ] | <u,U> | "poo,put" |
    | /e/ | [e]   | <e>   | "pay"     || /o/ | [o]   | <o>   | "oat"     |
    | /ɛ/ | [ɛ]   | <E>   | "pet"     || /ɔ/ | [ɔ]   | <O>   | "oar"     |
    | /a/ | [a]   | <a>   | "pat"     || /ɒ/ | [ɒ,ɑ] | <Q,A> | "pot,art" |

\section{Allophony}

/l/

\section{Accents}

Some accents contain ejectives

Some exhibit velarisation and/or palatisation

\section{Phonotactics}

Velars seem to appear predominantly as codas

Voiced stops rarely occur in a coda position

Nuclei can be composed of vowels, nasals, fricatives (without secondary articulation), and /l/

Single syllable morphemes require a vowel or fricative nucleus

% Misc
% ----
% The lexicon appears to be prefix free?

\part{Lexicon}
\index{Lexicon}

\part{Corpus}

\backmatter
\printindex
\end{document}
